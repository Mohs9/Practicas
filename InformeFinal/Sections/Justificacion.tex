\addcontentsline{toc}{chapter}{Justificación}


\chapter*{Justificación del proyecto}


%\rrnote{Cambié de de orden los párrafos y los adapté }

En este trabajo se profundizará en el entendimiento de los operadores de densidad, operaciones cuánticas y su representación como suma de operadores, puesto que será útil para obtener una descripción completa del proceso de las mediciones difusas sobre un estado cuántico inicial. El análisis de los efectos de las mediciones difusas de un observable en un estado cuántico, ofrecerá una mirada integral sobre el proceso de medición en sistemas físicos de varias partículas a fin de adquirir una mejor comprensión de la teoría física.% fundamentos teóricos de un área de investigación en información cuántica.

Este proyecto permite desarrollar la teoría física, las habilidades y herramientas matemáticas obtenidas a lo largo de los cursos de la licenciatura. Asimismo, permite entender la teoría cuántica desde una prespectiva diferente y propicia la adquisición de nuevas destrezas para la resolución de problemas, que son necesarias para el cumplimiento de los objetivos mencionados en la sección anterior.



%\cpnote{Reestructura para que este parrafo sea el primero (te toca reescribirlo y adaptarlo). También porfa redacta como una justificación. Algo a lo largo de la ultima frase. }
 
  
  
 
  
 Y finalmente, este proyecto será útil para incursionar, en el campo de la investigación de la física, en particular en el área de la información cuántica. Además será una fase previa al trabajo de tesis de graduación que concluye la formación de un estudiante de licenciatura.    
