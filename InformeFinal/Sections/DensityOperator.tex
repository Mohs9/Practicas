\chapter{Operador de densidad}\label{OpDensidad}

El operador de densidad es una herramienta en mecánica cuántica que describe el estado cúantico de un sistema físico. Este operador también conocido como <<matriz de densidad>> es una generalización de los vectores de estado, puesto que estas matrices pueden representar estados mixtos. Además permite el cálculo de las probabilidades de los resultados de cualquier medición realizada sobre este sistema.  En la primera sección de este capítulo, se presenta esta herramienta, comenzando por la definición y motivación de este operador. Luego, en la segunda sección  se presenta las propiedades y reformulación de los postulados de la mecánica cuántica utilizando este operador. Por último, en la tercera sección se muestran una de las aplicaciones más importantes del operador de densidad. 



\section{Definición y motivación  del operador de densidad}
El lenguaje de los operadores de denisdad proporciona un  medio conveniente para describir los sistemas cuánticos, tales que su estado no es completamente conocido. Al cosiderar un sistema cuántico que se encuentra en alguno de los estados $|\psi_j \rangle $ con probabilidad $p_j$. El conjunto de todos los posibles estados conforman un ensamble
de estados puros del sistema, con lo cual se puede enunciar la siguiente definición. 

  \begin{definition}[\textbf{Operador de densidad}] El operador de densidad $\rho$ correspondiente al ensamble de estados puros $\{p_j,|\psi_j \rangle \}$ es definido como {\cite{wilde2011classical}}
  	\begin{equation}{\label{defdensityoperator}}
  		\rho=\sum_{j}p_j|\psi_j\rangle \langle \psi_j|
  	\end{equation}
  
  	\end{definition}
  	

 A menudo se utiliza el término matriz de densidad para referirse al operador de densidad, en la práctica estos dos términos se utilizan indstintamente. Esta formulación, es equivalente a la aproximación del vector de estado, sin embargo a veces es mucho más fácil acercarse a problemas desde este nuevo punto de vista.
 
 Considerese una medición en un ensamble, de algún observable $\mathcal{O}$. Una pregunta importante sería, ¿cuál es el valor promedio de $\mathcal{O}$ cuando se reproduce un número muy grande de mediciones? La respuesta está dada por el promedio $\langle \mathcal{O} \rangle$ del ensamble, el cual se define por {\cite{sakurai2017modern}}
 
 \begin{equation}
 	\label{expectationvalue}
 	 \begin{split}
 		\langle \mathcal{O} \rangle &= \sum_{i}p_i \langle\psi_i|\mathcal{O}|\psi_i\rangle\\
 		&=\sum_i\sum_{j}\sum_k \langle\psi_i||\phi_j\rangle \langle \phi_j|\mathcal{O}|\phi_k\rangle \langle \phi_k||\psi_i\rangle\\
 		&=\sum_{j}\sum_k \left(\sum_{i} p_i \langle \phi_k|\psi_i\rangle  \langle\psi_i|\phi_j\rangle\right) \langle \phi_j|\mathcal{O}|\phi_k\rangle, 
 	\end{split}
 \end{equation}
 donde $\{|\phi_j\rangle \}$ es una base más general del espacio de Hilbert del sistema. La última línea en la ecuación {\ref{expectationvalue}} es una motivación para  definir el operador de densidad  en la forma de la ecuación {\ref{defdensityoperator}}. El operador de densidad contiene toda la información físicamente importante que se puede obtener acerca del ensamble {\cite{sakurai2017modern}} 
 
 
 
  \begin{equation}
 	\label{expectationvalue_traza}
 	\begin{split}
 		\langle \mathcal{O} \rangle &=\sum_{j}\sum_k \langle \phi_k|\rho|\phi_j\rangle \langle \phi_j|\mathcal{O}|\phi_k\rangle\\
 		&= \rm{tr}(\rho \mathcal{O}).
 	\end{split}
 \end{equation}

	Además las mediciones se pueden describir fácilmente utilizando matrices de densidad. Si se realiza una medición descrita por lo operadores $M_m$ y con unestado inicial $|\psi_i\rangle$, luego la propabilidad de obtener el resultado $m$, dado el estado $i$ es $p(m|i)=\langle \psi_i|M_m^{\dagger}M_m|\psi_i\rangle = \tr(M_m^{\dagger} M_m |\psi_i\rangle \langle\psi_i|)$. Por la ley de probabilidad total\footnote{Si  $X$ y $Y$ son dos variables aleatorias, entonces las probabilidades para $Y $ pueden expresarse en términos de las probabilidades para $X$, y las probabilidades condicionales para $ Y$ dada $X$,
		\[p(y) =\sum_x p(y|x)p(x),\] donde la suma sobre los valores de $x$ que $X$ puede tomar {\cite{nielsen_chuang_2010}}.}, la probabilidad de obtener un resultado $m$ es {\cite{nielsen_chuang_2010}}


\begin{equation}
	\label{probabilidad_m}
	\begin{split}
	p(m) &= \sum_{i}p(m|i)p_i\\
	&=\sum_i p_i\tr(M_m^{\dagger} M_m |\psi_i\rangle \langle\psi_i|)\\
	&=\sum_{i}\tr(M_m^{\dagger} M_m \rho)
	\end{split}
\end{equation}
	


y el estado después de la medición será $|\psi_i^m\rangle=\frac{M_m|\psi_i\rangle}{\sqrt{\langle\psi_j|M_m^{\dagger}M_m|\psi_i\rangle}}$,por lo que el operador de densidad correspondiente al ensamble de estados $|\psi_i^m\rangle$ con probabilidades $p(i|m)$ estará dado por

\[\rho_m=\sum_i p(i|m)\frac{M_m|\psi_i\rangle \langle \psi_i|M^{\dagger}}{\langle\psi_j|M_m^{\dagger}M_m|\psi_i\rangle}\]

y  utilizando el teorema de Bayes, para $p(i|m)=\frac{p(m|i)p_i}{p_m}$ se obtiene finalmente que el operador de densidad $\rho_m$ es{\cite{nielsen_chuang_2010}}

\begin{equation}
	\rho_m=\sum_i p_i \dfrac{M_m|\psi_i\rangle \langle \psi_i|M^{\dagger}}{\tr(M^{\dagger}_m M_m \rho )}=\dfrac{M_m \rho M_m^{\dagger}}{\tr (M^{\dagger}_m M_m\rho )}
\end{equation}


Ahora si, por alguna razón, el registro de la medición se pierde. Se tendría un sistema cuántico en el estado $\rho_m$ con probabilidad $p(m)$, pero ya no se conocería el valor real de $m$. El estado del sistema cuántico estaría descrito por el operdaor densidad siguiente

\begin{equation}
	\rho'=\sum_m p(m)\rho_m=\tr(M^{\dagger}_m M_m\rho )\dfrac{M_m\rho M_m^{\dagger}}{\tr (M^{\dagger}_m M_m \rho )}=M_m\rho M_m^{\dagger}
\end{equation}


 
 Ahora, supongase que la evolución de un sistema cerrado es descrito por el operador $U$. Si el sistema inicial se encuentra en el estado $|\psi_i\rangle$ con probabilida $p_i$ luego de la evolución, el sistema se encontrará en el estado $U|\psi_i\rangle$ con probailidad $p_i$. Luego, la evolución del operador de densidad está descrita por {\cite{nielsen_chuang_2010}} \[\rho=\sum_{i}p_i|\psi_i\rangle \langle \psi_i|\xrightarrow{U}\rho'=\sum_{i}Up_i|\psi_i\rangle \langle \psi_i|U^{\dagger}=U\rho U^{\dagger}.\]


Hasta el momento se puede ver con estas motivaciones que los postulados básicos de la mecánica cuántica ralacionados con la medición y la evolución unitaria pueden ser reformulados en el lenguaje de los operadores de densidad. En la siguiente sección se profundiza esta reformulación de los postulados.


\section{Propiedades del operador de densidad}\label{postulates}

En esta sección se desarrolla algunas de las caractéristicas y propiedades de los operadores de densidad. Así como se presenta la formulación de los postulados de la mecánica cuántica. 

Para iniciar se enuncia la siguiente proposición y el siguiente teorema. 

\begin{proposition}El operador de densidad es un operador hermítico.
	
\end{proposition}


\begin{proof}
	\begin{equation}
		\begin{split}
			\rho^\dagger&={\left(\sum_{i} p_i|\psi_i\rangle \langle \psi_i|\right)}^{\dagger}.\\
			&=\sum_{i} p_i {\left(|\psi_i\rangle \langle\psi_i|\right)}^{\dagger}\\
			&=\sum_{i} p_i |\psi_i\rangle \langle\psi_i|\\
			&=\rho.
		\end{split}
	\end{equation}
	
\end{proof}

De acuerdo a Nielsen y Chuang {\cite{nielsen_chuang_2010}}:

\begin{theorem}[\textbf{Caracterización del operador de densidad}] Un operador $\rho$ es un operador de densidad, aosciada a algún ensamble $\{p_i, |\psi_i\rangle\}$ si y solo si este satisface las siguientes condiciones:
\begin{enumerate}
	\item $\rho$ tiene una traza igual a uno.
	\item $\rho $ es un operador \textit{positivo semidefinido}
\end{enumerate}	
\end{theorem}


Ahora se procede a demostrar formalmente este teorema


\begin{proof}
	Primero se supondrá que $\rho $ es un operador de densidad por lo que debe cumplir con la definición {\ref{defdensityoperator}}, por lo que al computar la traza se obtiene que
	
	
	
	\begin{equation*}
		\begin{split}
			{\rm tr(\rho)}&=\tr\left(\sum_{i} p_i|\psi_i\rangle \langle \psi_i|\right) \\
			&=	\sum_{i}p_i \tr(|\psi_i\rangle \langle \psi_i|)\\
			&=\sum_{i}p_i \sum_j\langle \psi_j|\psi_i\rangle \langle \psi_i|\psi_j\rangle\\
			&=\sum_{i}\sum_j p_i \delta_{ij} \delta_{ij}\\
			&=\sum_i p_i=1\\
		\end{split}
	\end{equation*}

Ahora para la segunda condición, se toma un vector de estado arbitrario $|\varphi \rangle$

 
\begin{equation*}
	\begin{split}
	\langle \varphi |	\rho|\varphi \rangle&=\sum_{i}p_i\langle \varphi |\psi_i\rangle \langle \psi_i|\varphi \rangle \\
	&=\sum_{i}p_i| \langle \psi_i|\varphi \rangle|^2 \ge 0,\\
	\end{split}
\end{equation*}

La desigualdad se sigue porque cada $p_i$ es una probabilidad y por lo tanto no es negativa. Por lo que por definición de operadores positivos semidefinidos, $\rho$ cumple la segunda condición.


Ahora se debe mostrar el converso del enunciado anterior. Suponiendo $\rho$ es un operador que cumple las dos condiciones, entonces se debe demostrar que es un operador de densidad.

Dado que $\rho$ es un operador positivo semidefinido entonces tiene una descomposición espectral dada por la siguiente ecuación {\cite{nielsen_chuang_2010}}
\[\rho=\sum_i \lambda_i |i\rangle \langle i|\]

con $|i\rangle$ vectores ortogonales y $\lambda_i$ son valores propios reales no negativos, de la matriz $\rho$. Además satisface que la traza es uno 
\[\tr(\rho)=1=\sum_i\lambda_i\]

luego, un sistema estado $|i\rangle$ con una probabilidad $\lambda_i$ tendrá un operador asociado $\rho$. Esto significa que, el ensamble $\{\lambda_i, |i\rangle\}$  corresponde al operador $\rho$ que cumple con {\ref{defdensityoperator}}.


\end{proof}



Este teorema  permite obtener una definición equivalente del operador de densidad y con ello es posible reformular los postulados de la mecánica cuántica utilizando este operador. Nielsen y Chuang {\cite{nielsen_chuang_2010}}presentan la siguiente reformulación.


\setlength{\leftskip}{1cm}

 \textbf{Postulado 1:} Cualquier sistema físico tiene asociado un espacio de Hilbert $\mathcal{H}$, conocido como el \textit{espacio de estado} del sistema. El sistema está completamente descrito por su \textit{operador de densidad}, el cual es un operador $\rho$ positivo semidefinido con traza uno, actuando en el espacio de estado del sistema. Si un sistema cuántico está en el estado $\rho_i$ con una probabilidaf $p_i$, entonces el operador de densidad para el sistema es $\sum_{i}p_i\rho_i$.


\textbf{Postulado 2:} La evolución de un sistema cuántico cerrado está descrita por una \textit{transformación unitaria}. Esto es, el estado $\rho$ del sistema en el tiempo $t_1$ está relacionado con el estado $\rho'$ del sistema en el tiempo $t_2$ por un operador unitario $U$ que depende solamente de $t_1$ y $t_2$, 

\begin{equation}\label{postulado 2}
\rho'=U\rho U^{\dagger}
\end{equation}


\textbf{Postulado 3:} Las mediciones cuánticas están descritas por una colección $\{M_m\}$ de \textit{operadores de medición}. Estos operadores actúan en el espacio de estado del sistema que se está midiendo. El índice $m$ se refiere a las salidas de la medición que pueden ocurrir en el experimento. Si el estado del sistema cuántico es $\rho$ inmediatamente antes de la medición, luego la probabilidad de obtener el resultado $m$ está dada por

\begin{equation}\label{probaility3postulate}
	p(m)=\tr(M^\dagger M \rho)
\end{equation}


y el estado del sistema después de la medición es 


\begin{equation}\label{state3postulate}
	\rho_m'=\dfrac{M_m\rho M_m^\dagger}{\tr(M_m^\dagger M_m \rho)}.
\end{equation}


Los operadores de medición satisfacen la relación de completitud.

\begin{equation}\label{completitud3postulate}
	  	\sum_m M_m^\dagger M_m=\mathds{1}.
\end{equation}

\textbf{Postulado 4:} El espacio de Hilbert de un sistema físico compuesto es el producto tensorial de los espacios de Hilbert individuales de cada uno de los sistemas que componen al sistema total. Es decir, si el sistema total se compone de $N$ subsistemas, entonces
\begin{equation}\label{Htotal4postulado}
	\mathcal{H}_{\text{total}}=\mathcal{H}_1\otimes \mathcal{H}_2\otimes \cdots \otimes \mathcal{H}_N.
\end{equation}
  Más aún, si el sistema número $i$, con $i=1,2,\ldots,N$, está preparado en el estado $\rho_i$, luego el estado del sistema total será 

\begin{equation}\label{rhototal4postulado}
	\rho_{\text{total}}=\rho_1\otimes \rho_2 \otimes \cdots \rho_N
\end{equation}


\setlength{\leftskip}{0pt}

Con esta reformulación de los postulados de la mecánica cuántica se tiene la ventaja que es más fácil trabajar en la descripición de sistemas cuánticos cuyo estado no se conoce y la descripción de subsistemas de un sistema cuántico compuesto.


Ahora, será útil introducir algunos conceptos y hechos sobre operadores de densidad. Un sistema cuántico cuyo estado es conocido exactamente, se dice que se encuentra en \textit{estado puro}. En este caso, el operador de densidad es simplemente $\rho=|\psi \rangle \langle \psi|$. De lo contrario, $\rho$ está en un \textit{estado mixto}\footnote{En algunas referencias, se  usa el término <<estado mixto>> para incluir ambos estados cuánticos puros y mixtos. El origen de este uso puede ser que no se asume necesariamente que un estado es puro. Y además, el término <<estado puro>> se usa a menudo en referencia a un vector de estado $|\psi\rangle $, para distinguirlo de un operador de densidad	$\rho$. }; se dice que es una mezcla de los diferentes estado puros en el ensamble de $\rho$ {\cite{nielsen_chuang_2010}}. 

Estos términos dan pie a incluir las siguientes definiciones {\cite{wilde2011classical}}:

\begin{definition}[\textbf{Estado de máxima mezcla}] El estado de máxima mezcla $\sigma$ es el operador de densidad correspondiente a un conjunto uniforme de estados ortogonales $\{\frac{1}{d},|x\ra\}$, donde $d$ es la dimensionalidad del espacio de Hilbert. El estado de máxima mezcla $\sigma$ es entonces igual a
	
	
	\begin{equation}\label{maximallyMixedState}
		\sigma \equiv \dfrac{1}{d}\sum_x|x\rala x|=\dfrac{\mathds{1}}{d}.
	\end{equation}
\end{definition}


\begin{definition}[\textbf{Pureza}] La pureza $P(\rho)$ de un operador de densidad $\rho$ se define como sigue {\cite{wilde2011classical}}
\begin{equation}
	P(\rho)\equiv\tr(\rho^\dagger\rho)=\tr(\rho^2)
\end{equation}

	\end{definition}

La pureza es una medida particular del ruido de un estado cuántico.

\begin{proposition}
	La pureza de un operador de densidad es igual a uno si y solo si es un estado puro, y la pureza de un operador de densidad es estrictamente menor que uno si y solo si es un estado mixto.
	\end{proposition}


\begin{proof}
	Sea un operador de densidad dado por a siguiente ecuación, \[\rho=\sum_i p_i |\psi_i\rangle \langle \psi_i|\]
	
	con $p_i\ge 0$ y $\sum_{i}p_i=1$, se tiene que:
	
	\begin{equation*}
		\begin{split}
			\rho^2&=\sum_{i,j} p_i p_j|\psi_i\rala \psi_i|\psi_j\rala \psi_j|\\
			&=\sum_{i,j} p_i p_j|\psi_i \rala \psi_j|\delta_{ij}\\
			&=\sum_i p_i^2|\psi_i \rala \psi_i|
			\end{split}
	\end{equation*}
	Luego se obtiene la pureza,
	
	\[\tr(\rho^2)=\tr{\left(\sum_i p_i^2|\psi_i \rala \psi_i|\right)}=\sum_i p_i^2 \tr(|\psi_i \rala\psi_i|)=\sum_i p_i^2\]
	
	Ahora, dado que $p_i^2 \le p_i $, puesto que $0 \le p_i \le 1$. Luego, se tiene que \[\sum_i p_i^2\le\sum_i p_i=1.\]
	
   Y dado que $p_i^2<p_i$ para todo $p_i<1$, la igualdad se mantiene si y solo si la sumatoria tiene un único $p_i$ que debe ser igual a 1. Luego 	$\tr(\rho^2)=\sum_{i}p_i^2=\sum_i p_i= p_i=1$ entonces $\rho=|\psi_i\rala\psi_i|$ es puro. Y la pureza entonces será estrictamente menor a uno  $\tr(\rho^2)=\sum_{i}p_i^2<\sum_i p_i=1$ entonces todo $p_i<1$ y $\rho=\sum_i p_i |\psi_i\rangle \langle \psi_i|$ es mixto.
    
     
	Conversamente si $\rho $ es puro, luego $\rho=|\psi\rala \psi|$.
	
	\[\tr(\rho^2)=\tr(|\psi\rala \psi|\psi\rala \psi|)=\tr(|\psi\rala \psi|)=\la\psi|\psi\ra=1\]
	
	y si $\rho $ es mixto, luego $\rho=\sum_i p_i|\psi_i\rangle \langle \psi_i|$, con todo $p_i<1$.
	
	
	
	\[\tr(\rho^2)=\sum_i p_i^2 \tr(|\psi_i \rala\psi_i|)=\sum_i p_i^2<\sum_i p_i=1.\]
	\end{proof}



Antes de pasar a la siguiente sección, es indispensable discutir qué clase de ensambles pueden dar una matriz de densidad particular y cuándo dos conjuntos de vectores $\{|\psi_i\ra\}$ y$ \{|\phi_j\ra \}$ generan el mismo operador de densidad. La respuesta a estas interrogantes tiene muchas aplicaciones en información y computación cuántica. Y para ello se presenta el siguiente teorema que se enuncian Nielsen y Chuang {\cite{nielsen_chuang_2010}}




\begin{theorem}[\textbf{Libertad unitaria en el ensamble para matrices de densidad}]
Los conjuntos $\{|\psi_i\ra\}$ y$ \{|\phi_j\ra \}$, no necesariamente normalizados, generan la misma matriz de densidad si y solo si
\begin{equation}\label{teorema2.4}
|\psi_i\ra= \sum_{j}u_{ij}|\phi_j\ra
\end{equation}

donde $u_{ij}$ es una matriz unitaria compleja, con índices $i$ y $j$. Se completa con vectores 0 adicionales en caso uno de los conjutos tenga menos elementos.
\end{theorem}

\begin{proof}
	
	Primero, se supone que $|\psi_i\ra= \sum_{j}u_{ij}|\phi_j\ra$ para alguna matriz $u_{ij}$. Luego
	
	
	\begin{equation*}
		\begin{split}
			\sum_{i}|\psi_i\ra \la \psi_i|&=\sum_{ijk}u_{ij}u_{ki}^*|\phi_j\ra \la \phi_k|\\
			&=\sum_{jk}\left(\sum_i u_{ki}^\dagger u_{ij}\right) |\phi_j\ra \la \phi_k|\\
			&=\sum_{jk}\delta_{jk}|\phi_j\ra \la \phi_k|\\
			&=\sum_{j} |\phi_j\ra \la \phi_j|\\
	\end{split}
	\end{equation*}
	
	con lo que se prueba que $\{|\psi_i\ra\}$ y$ \{|\phi_j\ra \}$ generan el mismo operador.
	
	
	
	Conversamente, se supone que ambos conjuntos generan el mismo operador
	
	\[Q=\sum_{i}|\psi_i\ra \la \psi_i|=\sum_{j} |\phi_j\ra \la \phi_j|\]
	
	Luego la  descomposición espectral del operador es $Q = \sum_k \lambda_k |k \rala k|$ tal que los vectores $ |k\ra$ son ortonormales, y $\lambda_k$ son estrictamente  positivos. Sea $|\psi\ra$  un vector ortonormal al espacio generado por $|\tilde{k}\ra=\sqrt{\lambda_k}|k\ra$, por lo que $\la \psi|\tilde{k}\ra \la \tilde{k}|\psi\ra=0$ para todo $ k$, y luego se ve que
	
	\[0=\la \psi |Q|\psi\ra=\sum_{i}\la\psi|\psi_i\rala \psi_i|\psi \ra =\sum_i |\la\psi|\psi_i\ra|^2 \]
	
	y $\la\psi|\psi_i\ra=0$ para todo $i$ y todo $|\psi \ra$ ortonormal al espacio generado  por el $|\tilde{k}\ra$. Esto se sigue que cada $|\psi_i\ra $ puede expresarse como una combinación lineal de los vectores  $|\tilde{k}\ra$, $|\psi_i\ra=\sum_k c_{ik} |\tilde{k}\ra$. Como  $Q=\sum_k  |\tilde{k}\rala \tilde{k}|=\sum_{i}|\psi_i\rala\psi_i|$, se ve
	\[\sum_k |\tilde{k}\rala \tilde{k}|=\sum_{kl} \left(\sum_i c_{ik}c_{il}^{*}\right)|\tilde{k}\rala \tilde{l}|\]
	
	los operadores $|\tilde{k} \rala \tilde{l}|$ es fácil ver  que son linealmente independientes y debe ser que $\sum_i c_{ik}c_{il}^{*}=\delta_{kl}$. Esto asegura que se puedan agregar columnas extra a $c$ para obtener una matriz $v$ tal que $|\psi_i\ra=\sum_{k}v_{ik}|k\ra  $, donde se han agregado  vectores cero a la lista de $|\tilde{k}\ra$. Similarmente, se puede encontrar una matriz unitaria $w$ tal que $|\phi_j\ra=\sum_{k}w_{jk}|\tilde{k}\ra$. Luego $|\psi_i\ra=\sum_j u_{ij}|\phi_j\ra$, donde  $u=vw^\dagger$ es unitaria. 
	\end{proof}

\section{El operador de densidad reducido}

Una de las aplicaciones más importantes del operador de densidad es el operador de densidad reducido que es una herramienta descriptiva para subsistemas de un sistema cuántico. La matriz de densidad reducida es tan útil que es indispensable en el análisis de sistemas cúanticos compuestos y será discutida en esta sección.


Supóngase que se tiene sistemas $A$ y $B$, cuyos estados están descritos por el operador de densidad $\rho_{AB}$. El operador de densidad reducido o local para el sistema $A$ es definido por 
\begin{equation}
	\rho_A=\tr_B(\rho_{AB})
\end{equation}

donde $\tr_B$ es un mapeo de operadores conocidos como la traza parcial sobre el sistema $B$. Y la traza parcial tiene la siguiente definición, presentada por Wilde {\cite{wilde2011classical}}.





\begin{definition}[\textbf{Traza parcial}]
	
	Sea  $ \rho_{AB}$ un operador cuadrado actuando en un producto tensorial del espacio de Hilbert $\mathcal{H}_A \otimes \mathcal{ H}_B$ y sea $\{|l\ra_b\}$ una base ortonormal para el espacio $\mathcal{H}_B$. Luego la traza parcial sobre el sistema $\mathcal{H}_B$ está definida como sigue:
	
	\begin{equation}
		\tr_B(\rho_{AB})\equiv\sum_{l} (\mathds{1}\otimes\la l|_B)\rho_{AB}(\mathds{1}\otimes|l\ra_B).	
	\end{equation}

Por simplicdad, no se suele escribir los operadores identidad y se escribe de la siguiente forma:

\begin{equation}
	\tr_B(\rho_{AB})\equiv\sum_{l} \la l|_B\rho_{AB}|l\ra_B.
\end{equation}

	\end{definition}


Por la misma razón que la definición de la traza es invariante bajo la elección de una base ortonormal, lo mismo es cierto para la operación de traza parcial. También se observa, a partir de la definición anterior, que la traza parcial es una operación lineal.


En conclusión, dado un operador $\rho_{AB}$ que describe un estado conjunto de los sistemas $A$ y $B$, es posible calcular un operador de densidad local $\rho_A$, que describe el estado local de $A$ si el sistema $B$ es inaccesible para $A$.


Según Wilde {\cite{wilde2011classical}}, existe una forma alternativa de describir la traza parcial, la cual es útil estar consciente. Para un estado simple de la forma 
\begin{equation}
	|x_1\rala x_2|_A\otimes|y_1\rala y_2|_B
\end{equation}

la acción de la traza partial es como sigue:

\begin{equation}
	\begin{split}
	\tr_B(	|x_1\rala x_2|_A\otimes|y_1\rala y_2|_B)&=|x_1\rala x_2|_A\tr(|y_1\rala y_2|_B)\\
	&=|x_1\rala x_2|_A \la y_2|y_1\ra
	\end{split}
\end{equation}

donde se calcula la traza del segundo sistema para obtener el operador de densidad local del primero. Esto se puede generalizar para un operador de densidad arbitrario $\rho_{AB}$, con una base ortonormal $\{|i \ra_A \otimes |j\ra_B \} _{i,j}$ para un estado conformado por dos sitemas:


\begin{equation}
	\begin{split}
	\rho_{AB}&=\sum_{i,j,k,l}\lambda_{i,j;k,l}(|i\ra_A \otimes|j\ra_B)(\la k|_A\otimes \la l|_B)\\
	&=\sum_{i,j,k,l}\lambda_{i,j;k,l} |i\rala k|_A\otimes|j\rala l|_B
	\end{split}
\end{equation}
los coeficientes $\lambda_{i,j;k,l}$  son los elementos de la matriz $\rho_{AB}$ en la base respectiva.


luego al comptar la traza parcial y usando la linealidad de la traza, se obtiene


\begin{equation}
	\begin{split}
		\rho_A&=\tr_B\left(\sum_{i,j,k,l}\lambda_{i,j;k,l} |i\rala k|_A\otimes|j\rala l|_B\right)\\
		&=\sum_{i,j,k,l}\lambda_{i,j;k,l} \tr_B(|i\rala k|_A\otimes|j\rala l|_B)\\
		&=\sum_{i,j,k,l}\lambda_{i,j;k,l}|i\rala k|_A \tr(|j\rala l|_B)\\
		&=\sum_{i,j,k,l}\lambda_{i,j;k,l}|i\rala k|_A \la l|j\ra \\
		&=\sum_{i,j,k}\lambda_{i,j;k,j}|i\rala k|_A\\
		&=\sum_{i,k}\left(\sum_j\lambda_{i,j;k,j}\right)|i\rala k|_A.
	\end{split}
\end{equation}



Con el operador de densidad reducido se finaliza el marco teórico del lenguaje del operador de densidad necesario para el objetivo de este proyecto. En la siguiente sección, se estará hablando sobre operadores POVM que son de interés en este trabajo.



