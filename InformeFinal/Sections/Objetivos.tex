\addcontentsline{toc}{chapter}{Objetivos}

\chapter*{Objetivos}

\section*{Objetivo General}

Establecer los operadores de Kraus para mediciones difusas en un sistema de dos partículas. 


\section*{Objetivos Específicos}
\begin{itemize}
	\item Comprender la definición y propiedades de la matriz de densidad, así como los postulados de la mecánica cuántica utilizando el lenguaje de los operadores de densidad.
	
	\item Entender la acción de operaciones cuánticas en un operador de estado como la representación de suma de operadores de Kraus. 

%\rrnote{Reformulé este objetivo de nuevo:}
	\item Analizar las medidas POVM y su descomposición como operadores de Kraus que describen mediciones difusas en un sistema de dos partículas.
	
%\cpnote{Un observable de dos niveles degenerado es como la identidad y por consiguiente aburrido. Rereformula}
	
	\item Realizar un programa que mapee un operador de estado inicial a un estado final después de una medición difusa.

	
	
\end{itemize}
