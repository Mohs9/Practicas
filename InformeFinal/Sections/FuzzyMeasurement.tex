\chapter{Mediciones difusas}


En el capítulo {\ref{MedidaPOVM}} y en la sección {\ref{Medicion_RepresentacionDeKraus}} se habló sobre medidas POVM y sobre mediciones. En este capítulo se presentan las mediciones difusa de sistemas cuánticos de dos partículas (que puede ser generalizado para más de dos partículas) en las que se puden identificar partículas individuales, sin embargo, siempre hay una probabilidad de identificarlas erróneamente. Estas detecciones imperfectas se exponen en la referencia {\cite{Pineda_2021}}. También se utiliza el lenguaje de las operaciones cuánticas por medio de los operadores de Kraus para poder describir los sistemas cuánticos de dos partículas.

La estructura de este capítulo es la siguiente. En la sección 4.1 se establece el problema de las mediciones difusas. Luego, en la sección 4.2, se discutirá el conjunto de operadores de Kraus que describan los efectos de las mediciones  
en estado de entrada del sistema para dos partículas, asimismo, que proporcionen la probabilidad de cada resultado de medición posible para cualquier estado inicial.


\section{Mediciones difusas para dos partículas}

Pineda, Davalos, Viviescas y Rosado, proponen el siguiente ejemplo. <<Una cadena de iomes se hace brillar  y se obtiene una señal fluorescente. Sin embargo, debido a la imperfecciones del detector, no es posible determinar
determinar el origen exacto de la señal fluorescente. La información obtenida en este caso se vuelve borrosa, pero su cuantificación aún es posible>>.

Si se  realiza una  medición en un sistema de muchas partículas, pero no se está seguro en cual partícula esta medición fue aplicada, se obtiene una medición difusa. En el caso de un sistema de dos partículas, en el cual se desea realizar una medición del observable $A\otimes B$, el dispositivo de medición confunde las partículas con una probabilidad $(1-p)$, y a veces en su lugar, realiza la medición de $B\otimes A$. Luego, si $\rho$ es el estado inicial del sistema, el valor esperado para la salida de la medición será {\cite{Pineda_2021}}  \[\la A\otimes B\ra_{\mathcal{F_{\text{2p}}[\rho]}}=\tr(\mathcal{F_{\text{2p}}[\rho]}A\otimes B)=p\tr(\rho A\otimes B)+(1-p)\tr(\rho B\otimes A),\]


donde el operador $F_{\text{2p}}[\rho]=p\rho+(1-p)S_{01}\rho$ 

