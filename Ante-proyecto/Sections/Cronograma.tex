\section{Cronograma}


\begin{table}[H]
	\begin{center}
		
	\begin{tabular}{|>{\arraybackslash}m{6cm}|>{\centering}m{1cm}|>{\centering}m{1cm}|>{\centering}m{1cm}|>{\centering}m{1cm}|>{\centering}m{1cm}|c|}
		\hline
		
	%\cellcolor[HTML]{C0C0C0} para agregar color
		\diagbox[width=18em]{  Tareas\\ }{ \\ Meses}&
		 Mes 1& Mes 2&Mes 3&Mes 4 &Mes 5&Mes 6 \\ \hline \hline
		\textbf{Tarea 1:} Repasar los conceptos de álgebra lineal que se ocupan en la mecánica cuántica y estudiar los postulados de la mecánica cuántica utilizando el lenguaje de la matriz de densidad.&\faCheck & & & & & \\ \hline
		\textbf{Tarea 2:} Estudiar los conceptos básicos del formalismo de las operaciones cuánticas.&&\faCheck&\faCheck&&& \\ \hline
		\textbf{Tarea 3:} Estudiar el espacio de las matrices de densidad.&&&\faCheck&&&\\ \hline
		\textbf{Tarea 4:} Estudiar el proceso de medición de observables. &                          & &  & \faCheck &  &                          \\ \hline
		\textbf{Tarea 5:} Analizar la ecuación (1) de la referencia \cite{Pineda_2021} y establecer los operadores de Kraus de la medición de un observable en el caso no degenerado en un sitema de dos partículas. &                          &  & &\faCheck   &  &                          \\ \hline
		\textbf{Tarea 6:}Establecer los operadores de Kraus de la medición para el caso degenerado y también en el que el observable no pueda escribirse como producto tensorial.&                          &                          &  & &\faCheck   &                          \\ \hline
		\textbf{Tarea 7:} Escribir un programa que realice una operación cuántica al operador de estado inicial. &                          &                          &  &  & \faCheck &                          \\ \hline
		\textbf{Tarea 8:} Elaboración del informe final.&                          &                          &  &  &  &\faCheck  \\ \hline
	\end{tabular}

	\end{center}

\end{table}