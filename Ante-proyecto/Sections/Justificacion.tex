\section{Justificación del proyecto}

	  Este trabajo permite desarrollar la teoría física, las habilidades y herramientas matemáticas obtenidas a lo largo de los cursos de la licenciatura. Asimismo, propicia la adquisición de nuevas destrezas para la resolución de problemas, que son necesarias para el cumplimiento de los objetivos mencionados en la sección anterior. 
	  
	  Específicamente, los operadores de densidad son una formulación matemática equivalente al enfoque del vector de estado que se estudia en los primeros cursos de mecánica cuántica de la licenciatura, por lo que estudiar estos operadores permite entender la teoría cuántica desde una prespectiva diferente. Así como aprender e investigar sobre operaciones cuánticas permitirá incrementar el conocimiento obtenido en dichos cursos.
	  
\rrnote{Agregué esta parte } 


Investigar acerca de las operaciones cuánticas y su representación como suma de operadores, será útil para obtener una descripción completa del efecto de las mediciones difusas sobre un estado cuántico inicial. Por otra parte, el análisis de los efectos de las mediciones difusas de un observable en un estado cúantico, es favorable para obtener una mejor comprensión del proceso de medición en sistemas físicos de varias partículas.

 
  
  
 
  
 Y finalmente, este proyecto será útil para incursionar, en el campo de la investigación de la física, en particular en el área de la información cuántica. Además será una fase previa al trabajo de tesis de graduación que conluye la formación de un estudiante de licenciatura.    
