\section{Descripción del grupo de trabajo}

El proyecto es dirigido por el Dr. Carlos Pineda y por el Ing. Rodolfo Samayoa.


\subsection{Ing. Rodolfo Samayoa}


El ingeniero Rodolfo Samayoa es egresado de la facultad de Ingeniería de la Universidad de San Carlos de Guatemala.
 Actualmente trabaja como profesor y jefe del Departamento de Física de en la Escuela de Ciencias Físicas y Matemáticas de la Universidad de San Carlos de Guatemala. Imparte los cursos de Mecánica Cuántica 1 y 2, así como otros cursos en el Departamento de Física y en el Departamento de Matemática de la ECFM.

\subsection{Dr. Carlos Francisco Pineda Zorrilla}
El doctor Carlos Pineda es investigador del Instituto de Física, desde el año de 2009. Se graduó como físico de la Universidad Nacional de Colombia con Grado de Honor, y posteriormente realizó sus estudios de posgrado en la UNAM. Seguidamente hizo un posdoctorado con Jens Eisert en la Universidad de Potsdam. Actualmente se desempeña como Investigador Titular B y tiene la distinción C del PRIDE y el nivel II del SNI. Sus intereses se han ampliado a los sistemas complejos. En colaboración con varios colegas desarrolló una herramienta para el estudio del rango y su evolución, la cual se ha aplicado de manera exitosa para estudiar la evolución varios sistemas complejos \cite{ifunam}.


Las líneas de investigación en las que trabaja el Dr. Pineda son:

\begin{itemize}
	\item Caos clásico.
	\item Decoherencia.
	\item Dinámica cuántica no markoviana.
	\item Información cuántica.
	\item Sistemas complejos.
	\item Teoría de matrices aleatorias.
	\item Teoría de muchos cuerpos.
\end{itemize}

