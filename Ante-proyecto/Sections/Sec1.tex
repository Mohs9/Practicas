\section{Descripción general de la institución}

\begin{comment}
\subsection{Escuela de Ciencias Físicas y Matemáticas - Universidad      de San	Carlos de Guatemala}


El Instituto de Investigación de Ciencias Físicas y Matemáticas (IFIM) es la unidad de la Escuela de Ciencias Físicas y Matemáticas de la Universidad de San Carlos (ECFM) que promueve y realiza estudios avanzados en áreas científicas, fundamentales y aplicadas, de las ciencias físicas y matemáticas.  El IFIM trabaja en la investigación de ciencia básica y aplicada. Se encarga de promover la investigación en estas en el ámbito universitario, de difundir y divulgar del conocimiento generado por la investigación en ciencias físicas y matemáticas, así como también de actualizar programas académicos de ciencias físicas y matemáticas.


Las principales áreas y líneas de investigación son: Física de altas energías, cosmología y astrofísica, Ciencias no lineales y sistemas complejos, Geofísica, Física experimental y Física de radiaciones.

\end{comment}



\subsection{Escuela de Ciencias Físicas y Matemáticas-Universidad de San Carlos de Guatemala}

La Escuela de Ciencias Físicas y Matemáticas (ECFM) de la Universidad de San Carlos de Guatemala tiene como fin que Guatemala pueda incorporarse al desarrollo de las ciencias físicas y matemáticas a través del refuerzo de los programas de grado y el fomento de los programas de postgrado para que pueda generarse dearrollo experimental a través de la investigación tanto básica como aplicada.

La ECFM busca promover la investigación en Física y Matemática con calidad mundial, al mismo tiempo que aumentar el número y calidad  de publicaciones a nivel internacional y local, participar en equipos multidisciplinarios de investigación, sobre todo en aquellos que busquen la solución de algunos de los problemas de la Universidad y de Guatemala en los que se pueda contribuir. 



\subsection{Instituto de Física - Universidad Nacional Autónoma de México (IFUNAM)}
El Instituto de Física (IFUNAM) es uno de los centros de investigación en física más importantes de México con un sólido prestigio internacional. En el IFUNAM se realiza una parte muy significativa de la investigación en física que se lleva a cabo en México, además se cultiva la docencia y formación de recursos humanos como actividades fundamentales.

El Instituto de Física ha publicado cerca de 6100 artículos, la mayoría en revistas de circulación internacional, además de otros múltiples productos de investigación. Asimismo ha jugado un papel relevante en la definición e implementación de políticas y programas científicos, no sólo a nivel nacional, sino con impacto en Latinoamérica.

 

Las actividades de investigación del Instituto de Física se llevan a cabo en ocho departamentos, los cuales son: Física Teórica, Estado Sólido, Materia Condensada, Física Experimental, Física Química, Sistemas Complejos, Física Cuántica y Fotónica y Física Nuclear y Aplicaciones de la Radiación. Además, el Instituto de Física cuenta con las siguientes áreas de investigación: física nuclear y de radiaciones; física médica; física atómica y molecular; materia condensada; óptica; física de materiales y nanociencias; sistemas complejos; física estadística; física biológica; física de partículas elementales; teoría de campos y cosmología; así como física y óptica cuántica. Los resultados de las investigaciones que se generan en estas áreas son difundidos en publicaciones de nivel internacional, seminarios y conferencias.




El proyecto es dirigido por el Dr. Carlos Pineda (IFUNAM) y por el Ing. Rodolfo Samayoa (ECFM).
